\documentclass[12pt]{article}
\usepackage[left=30mm,right=30mm,top=30mm,bottom=30mm]{geometry}
\usepackage{kotex}
\usepackage{amsmath}
\usepackage{setspace}
\usepackage[ruled]{algorithm2e}
\usepackage{enumitem}

\begin{document}
	\setstretch{1.25}
\begin{center}
	\LARGE \textsf{Homework 2 Report}\\
	\large \textsf{자연과학대학 화학부 $\cdot$ 2017--19871 남준오}\\
	\normalsize
	\vspace{1cm}
\end{center}

\noindent\textbf{1. 변경한 문법}
\setstretch{1.00}
\begin{align*}
	E & ~\rightarrow~ Te \\
	e & ~\rightarrow~ +Te ~|~ -Te ~|~ \epsilon \\
	T & ~\rightarrow~ Ft \\
	t & ~\rightarrow~ *Ft ~|~ /Ft ~|~ \epsilon \\
	F & ~\rightarrow~ (E) ~|~ A \\
	A & ~\rightarrow~ a ~|~ b ~|~ c ~|~ d ~|~ x ~|~ y ~|~ z ~|~ 2 ~|~ 3 ~|~ 4 ~|~ 5 ~|~ 6 ~|~ 7
\end{align*}
\setstretch{1.25}

\noindent\textbf{2. 파싱 테이블}

\begin{table}[!htbp]
	\begin{center}
	\renewcommand{\arraystretch}{1.2}
	\begin{tabular}{|c|c|c|c|c|c|c|c|c|}
		\hline
		{} & $+$ & $-$ & $*$ & $/$ & $($ & $)$ & $a^\dagger$ & $\#$ \\ \hline
		$E$ & & & & & $E \rightarrow Te$ & & $E \rightarrow Te$ & \\
		$e$ & $e \rightarrow +Te$ & $e \rightarrow +Te$ & & & & $e \rightarrow \epsilon$ & & $e \rightarrow \epsilon$ \\
		$T$ & & & & & $T \rightarrow Ft$ & & $T \rightarrow Ft$ & \\
		$t$ & $t \rightarrow \epsilon$ & $t \rightarrow \epsilon$ & $t \rightarrow *Ft$ & $t \rightarrow /Ft$ & & $t \rightarrow \epsilon$ & & $t \rightarrow \epsilon$\\
		$F$ & & & & & $F \rightarrow (E)$ & & $F \rightarrow A$ & \\
		$A$ & & & & & & & $A \rightarrow a$ & \\
		\hline
	\end{tabular}
	\end{center}
	$\dagger$ 여기서 $b, c, \cdots, 7$의 경우에는 $a$와 동일하고, $A \to a$만 각각에 대해 바꿔주면 된다.
\end{table}
\vspace{0.5cm}

\noindent\textbf{3. 테스트 케이스}\\
\begin{itemize}[nolistsep]
	\item Yes: \texttt{"(((((((((((((((((((((((((((((((((((((((((((((((((a)))))))))))))\\))))))))))))))))))))))))))))))))))))"} (괄호 49 쌍, 총 99 글자)\\
	
	입력 길이가 100 글자로 제한되어 있으며, 받아들여지는 글자는 항상 홀수개이므로 Yes의 출력을 내 놓는 최대 99 글자이다. 만약 문장 형태를 저장하는 문자열을 길이가 정해진 \texttt{char}형 배열으로 구현한다면, 길이가 긴 문장 형태를 만드는 입력은 오버플로우를 발생시키기 쉬울 것이다. 위 99 글자의 입력은 2번과 같은 파싱 테이블을 작성할 경우 최대 199 글자의 문장 형태를 만든다. 생략된(출력이 너무 길어 보고서에 모두 기록할 수 없음) 출력은 아래와 같다.\\
	
	\texttt{Yes\\
	E\\
	=> Te\\
	=> Fte\\
	=> (E)te\\
	=> (Te)te\\
	=> (Fte)te\\
	=> ((E)te)te\\
	=> ((Te)te)te\\
	$\vdots$\\
	=> (((((((((((((((((((((((((((((((((((((((((((((((((a)))))))))))))))\\))))))))))))))))))))))))))))))))))e\\
	=> (((((((((((((((((((((((((((((((((((((((((((((((((a)))))))))))))))\\))))))))))))))))))))))))))))))))))\\
	}
	\vspace{0.3cm}
	
	\item No: \texttt{"()"}\\
	
	$E$에서 시작하여 유도 과정이 끝나려면 결국에는 $A$에서 문자 형태가 되어야 한다. 따라서 적어도 $E$가 생성하는 문장은 $\epsilon$은 될 수 없다. 따라서 문자열 \texttt{"()"}은 $(E)$에서 $E$가 $\epsilon$이 될 수 없으므로 만들어질 수 없는 문장이다. 문법의 변경이나 파싱 테이블의 구현에 있어 이러한 부분에서 실수가 있었다면 Yes가 뜰 수도 있을 것이다. 출력은 다음과 같다.\\
	
	\texttt{No}
	
\end{itemize}


\end{document}​​​​​​​​​​​​​​​